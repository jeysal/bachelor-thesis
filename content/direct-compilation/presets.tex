\section{Presets}
With the direct compilation approach,
the two mocking library presets
\textit{@spockjs/preset-sinon-mocks} and
\textit{@spockjs/preset-jest-mocks}
each have to export an array of
\textit{interaction processors}.
An interaction processor has the following type signature:
\begin{minted}{typescript}
(
  t: typeof BabelTypes,
  config: InternalConfig
) => {
  primary: boolean;
  declare(
    interaction: InteractionDeclaration
  ): BabelTypes.Statement;
  verify(
    mockObject: BabelTypes.Expression
  ): BabelTypes.Statement;
};
\end{minted}

The signature somewhat resembles that of
the \textit{@spockjs/interaction-block} package itself
in that it also offers the
\textit{declare} and \textit{verify} operations.
This is because \textit{@spockjs/interaction-block},
aside from parsing interaction declarations,
mostly just delegates them through with this approach,
as we will see when finishing the implementation of that package soon.

The interaction processor property \textit{primary} indicates
whether the processor is meant for standalone use (\code{true}),
as is the case for those implementing mocking library bindings,
or it is an auxiliary processor (\code{false}),
such as one that provides better integration with a test runner.
Users will see an error when attempting to use interaction blocks
without a primary interaction processor.

The Sinon.JS and Jest mocking library presets
do not contain the corresponding interaction processors themselves.
Instead, like all such components of spockjs,
including the similar assertion post processors,
the interaction processors exist in the separate packages
\textit{@spockjs/interaction-processor-sinon-mocks} and
\textit{@spockjs/interaction-processor-jest-mocks},
which allows for them to be easily combined in other presets.
The \textit{@spockjs/preset-sinon-mocks} and
\textit{@spockjs/preset-jest-mocks} packages
simply import the interaction processors from those separate packages
and re-export them in single-element arrays.
