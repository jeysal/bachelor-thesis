\section{Sinon.JS}
As we will see,
Sinon.JS (\textit{Sinon}) provides a
quite rich stub and mock API
that we can leverage to make
compiling interaction blocks
to Sinon calls very straightforward.

On the other hand,
Sinon is peculiar in that it has a notion of
mock objects with methods on them
as an alternative to plain mock functions:
\autocite{SinonMockDoc}
\begin{minted}{javascript}
const obj = { method: () => {} };
const mock = sinon.mock(obj);
mock.expects("method").once().throws();
\end{minted}
If we see a member expression like
\code{obj.method()}
instead of just
\code{func()},
we can assume that it is such a mock object
and need to make sure that
we generate the proper \code{expects} call
like in the example above.
For plain mock functions,
we can skip the \code{expects}:
\begin{minted}{javascript}
mockFunc.once().throws();
\end{minted}

In the following,
we will focus on the most complex case:
a combined interaction declaration
(i.e. mocking and stubbing) with a
function that is a member of a mock object, e.g.:
\begin{minted}{javascript}
1 * mock.method(42) >> 1337;
\end{minted}
All the other cases with
stubbing/mocking only
or simple functions
instead of mock objects
are effectively just
boiled down versions of this case.
As usual, the full source code of
\textit{@spockjs/interaction-processor-sinon-mocks}
can be found in the appendix,
along with test cases that verify
the behavior of spockjs in conjunction with Sinon
for both implementation approaches.

\subsection{Declaration}\setminted{samepage=no}
The \textit{declare} function for this case starts by
checking whether the interaction declaration is of type \textit{combined} in line 4,
checking whether the mock object is a member expression in line 7, and
extracting the data from the declaration and member expression
in lines 2 and 5 and 8 to 12:
\begin{minted}[linenos]{typescript}
interaction => {
  const { mockObject, args } = interaction;

  if (interaction.kind === 'combined') {
    const { cardinality, returnValue } = interaction;

    if (t.isMemberExpression(mockObject)) {
      const {
        object: mock,
        property: method,
        computed
      } = mockObject;
\end{minted}

Before generating code,
we need to do one more thing.
Because the member expression could have a computed property name
(\code{a["b"]} instead of \code{a.b})
we cannot always simply wrap the name of the identifier
in a string literal and pass that
to the Sinon \textit{expects} function
as shown in lines 18 to 20.
In the case of a computed property name,
we instead stringify it and pass it to \textit{expects}
in lines 14 to 17:
\begin{minted}[linenos,firstnumber=last]{typescript}
      const methodName = computed
        ? t.callExpression(
            t.identifier('String'),
            [method]
          )
        : t.stringLiteral(
            (method as Identifier).name
          );
\end{minted}

Finally, we instantiate the Babel template
for a combined interaction declaration
with the nodes we have extracted and computed:
\begin{minted}[linenos,firstnumber=last]{typescript}
      return declareMockAndStubInteraction({
        MOCK: mock,
        METHOD_NAME: methodName,
        ARGS: args,
        CARDINALITY: cardinality,
        RETURN_VALUE: returnValue,
      });
    }
  }
}
\end{minted}

The corresponding template is:
\begin{minted}[linenos]{javascript}
const declareMockAndStubInteraction = template(`
  MOCK
    .expects(METHOD_NAME)
    .withArgs(ARGS)
    .atLeast(CARDINALITY)
    .atMost(CARDINALITY)
    .returns(RETURN_VALUE);
`);
\end{minted}
The other templates for the simpler cases
\begin{itemize}
  \item omit the \textit{expects} for plain mock functions,
  \item omit the \textit{atLeast} and \textit{atMost} for stub-only declarations, or
  \item omit the \textit{returns} for mock-only declarations.
\end{itemize}

We could generate the templates dynamically
instead of writing them out for all the cases,
but it does not appear to be worthwile for the small number of cases.
Also, mocks and stubs are considered different object types by Sinon,
so even though parts of the API --- like \textit{withArgs} --- are identical,
it would be awkward to treat them as the same thing
and just plug in function calls to our liking.

\setminted{samepage}\subsection{Verification}
We recorded all declarations by
converting them into a format understood by Sinon
and passing them to the Sinon mock object API.
Given that the mock object now already holds all the state
that is required to compare the expected and actual calls,
we can simply generate a call to the \textit{verify} function
of the mock object using the following template:
\begin{minted}{javascript}
const verify = template(`
  MOCK.verify();
`);
\end{minted}
