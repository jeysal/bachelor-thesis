\section{Sinon.JS}
As we will see,
Sinon.JS (\textit{Sinon}) provides a
quite rich stub and mock API
that we can leverage to make
compiling interaction blocks
to Sinon calls very straightforward.

\subsection{Templates}
In order to generate code that calls the Sinon API,
we will need some more tooling.
We have already created some AST nodes in other places such as identifiers,
and even non-leaf nodes such as call expressions,
but the chained calls to the fluent Sinon API
require \textit{a lot} of nodes.
Creating nodes via \code{t.identifier} and the like was fine
for the so far mostly small structures like
identifiers and simple call expressions,
but it would be massively cumbersome to handle
for larger tree structures.

This is where \textit{babel-template} comes into play.
\textit{babel-template} allows us to create AST nodes
by writing them as source code:
\autocite{BabelTemplateDoc}
\begin{minted}{javascript}
const buildRequire = template(`
  const IMPORT_NAME = require(SOURCE);
`);

const requireDecl = buildRequire({
  IMPORT_NAME: t.identifier('myModule'),
  SOURCE: t.stringLiteral('my-module'),
});
\end{minted}
Parts of the AST
--- in the example above, \code{IMPORT_NAME} and \code{SOURCE} ---
can be dynamically substituted with other nodes.
This capability is commonly known as
\textit{quasiquotes} or \textit{quasiquotation}.
\autocite{LispQuasiquotation}

\subsection{Code generation}
Sinon is peculiar in that it has a notion of
mock objects with methods of them
as an alternative to plain mock functions:
\autocite{SinonMockDoc}
\begin{minted}{javascript}
const obj = { method: () => {} };
const mock = sinon.mock(obj);
mock.expects("method").once().throws();
\end{minted}
If we see a member expression like
\code{obj.method()}
instead of just
\code{func()},
we assume that it is such a mock object
and need to make sure that
we generate the proper \code{expects} call
like in the example above.

In the following,
we will focus on the most complex case:
a combined interaction declaration
(mocking and stubbing) with a
function that is a member on a mock object, e.g.:
\begin{minted}{javascript}
1 * mock.method(42) >> 1337;
\end{minted}
All the other cases with
stubbing/mocking only
or simple functions
instead of mock objects
are effectively just
boiled down versions of it.
As usual, the full source code of
\textit{@spockjs/interaction-processor-sinon-mocks}
can be found in the appendix.
