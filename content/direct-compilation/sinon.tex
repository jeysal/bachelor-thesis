\section{Sinon.JS}
As we will see,
Sinon.JS (\textit{Sinon}) provides a
quite rich stub and mock API
that we can leverage to make
compiling interaction blocks
to Sinon calls very straightforward.

\subsection{Templates}
In order to generate code that calls the Sinon API,
we will need some more tooling.
We have already created some AST nodes in other places such as identifiers,
and even non-leaf nodes such as call expressions,
but the chained calls to the fluent Sinon API
require \textit{a lot} of nodes.
Creating nodes via \code{t.identifier} and the like was fine
for the so far mostly small structures like
identifiers and simple call expressions,
but it would be massively cumbersome to handle
for larger tree structures.

This is where \textit{babel-template} comes into play.
\textit{babel-template} allows us to create AST nodes
by writing them as source code:
\autocite{BabelTemplateDoc}
\begin{minted}{javascript}
const buildRequire = template(`
  const IMPORT_NAME = require(SOURCE);
`);

const requireDecl = buildRequire({
  IMPORT_NAME: t.identifier('myModule'),
  SOURCE: t.stringLiteral('my-module'),
});
\end{minted}
Parts of the AST
--- in the example above, \code{IMPORT_NAME} and \code{SOURCE} ---
can be dynamically substituted with other nodes.
This capability is commonly known as
\textit{quasiquotes} or \textit{quasiquotation}.
\autocite{LispQuasiquotation}
