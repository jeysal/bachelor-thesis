\section{Interaction processor integration}
We have created a parser for
interaction declaration blocks
in the previous chapter.
This parser already makes up the largest portion of the
\textit{@spockjs/interaction-block} package.
We can now implement the rest of the package.

The outer function,
having received the Babel \textit{types} helper
and the spockjs configuration object,
extracts the interaction processors from the configuration
and prepares them for use in a \code{processors} array.
It also validates that the config has at least one
primary interaction processor to make sure
the user did not forget to enable a preset for their mocking library:
\begin{minted}[fontsize=\scriptsize]{typescript}
export default (t: typeof BabelTypes, config: InternalConfig) => {
  const processors = config.hooks.interactionProcessors.map(processor =>
    processor(t, config),
  );
  if (!processors.some(({ primary }) => primary)) {
    throw new Error(/* ... */);
  }

  return {
    declare, // ...
    verify, // ...
  }
}
\end{minted}

The \textit{declare} function first ensures that
the interaction declaration block has the expected structure
of a binary expression statement.
If it does, the parser is called
to obtain the interaction declaration,
which is then passed to every
interaction processor.
The collective statement nodes returned from
the declaration functions of the processors
are used to replace the interaction declaration block.
The following code shows the \textit{declare} function implementation,
eliding error case handling:
\begin{minted}[fontsize=\scriptsize]{typescript}
(statementPath: NodePath<BabelTypes.Statement>) => {
  if (statementPath.isExpressionStatement()) {
    const expressionPath = statementPath.get('expression') as NodePath<
      BabelTypes.Expression
    >;

    if (expressionPath.isBinaryExpression()) {
      const declaration = parseInteractionDeclaration(expressionPath);
      statementPath.replaceWithMultiple(
        processors.map(({ declare }) => declare(declaration)),
      );
    }
  }
}
\end{minted}

Even more simply,
the \textit{verify} function just replaces the interaction verification block
with the collective statement nodes returned from
the verification functions of the processors
applied to the mock object expression:
\begin{minted}[fontsize=\scriptsize]{typescript}
(statementPath: NodePath<BabelTypes.Statement>) => {
  if (statementPath.isExpressionStatement()) {
    statementPath.replaceWithMultiple(
      processors.map(({ verify }) => verify(statementPath.node.expression)),
    );
  }
}
\end{minted}
