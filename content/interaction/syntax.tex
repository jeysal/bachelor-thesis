\section{Syntax}
Like assertion blocks, interaction blocks are also introduced by a labeled statement.
The label name may be either \textit{mock} or \textit{stub},
with no technical distinction being made between those two ---
although users are, of course, urgently reminded not to use them inappropriately.

The body of such a labeled statement
--- or, in the case of a block statement, each statement contained within it ---
may have the shape of a \textit{Mock-}, \textit{Stub-}, or \textit{CombinedInteractionDeclaration}
as shown in Figure~\ref{fig:InteractionBlockVariants}.
The multiplication and right shift operators are the same
that the Spock Framework uses for interaction declarations.

\begin{figure}
  {\color{teal}
    \begin{minted}{text}
MockInteractionDeclaration:
cardinality * call;
    \end{minted}
  }
  {\color{magenta}
    \begin{minted}{text}
StubInteractionDeclaration:
call >> returnValue;
    \end{minted}
  }
  {\color{violet}
    \begin{minted}{text}
CombinedInteractionDeclaration:
cardinality * call >> returnValue;
    \end{minted}
  }
  \caption{
    The source code structure of a Mock-, Stub-, or CombinedInteractionDeclaration
  }\label{fig:InteractionBlockVariants}
\end{figure}

Figure~\ref{fig:InteractionBlockAst} shows the corresponding AST structure
for all three variants, distinguished by color.
The terminal nodes we need to extract are
\begin{itemize}
  \item the \textit{mockObject},
  \item the \textit{args},
  \item the \textit{cardinality} (for mock interactions), and
  \item the \textit{returnValue} (for stub interactions).
\end{itemize}
The \textit{mockObject} and the \textit{args} always appear
in the same \code{CallExpression} \textit{call}
that is shown in Figure~\ref{fig:InteractionBlockCallAst},
although that call may be attached to three different sections of the AST.

\begin{figure}
  \centering \setminted{fontsize=\footnotesize}
  \forestset{
    qtree/.style={
      baseline,
      for tree={
        align=center,
        l=3cm,
      }}}
  \begin{forest}, baseline, qtree
    [ExpressionStatement
    [ BinaryExpression (operator: {\color{teal}\code{*}} / {\color{magenta}\code{>>}} / {\color{violet}\code{>>}}), edge label={node[midway,left]{\code{expression}}}
    [  {\color{teal}Expression \code{cardinality}}\\{\color{magenta}CallExpression \code{call}}\\{\color{violet}BinaryExpression (operator: \code{*})}, edge label={node[midway,left]{\code{left}}}
    [   {\color{violet}Expression \code{cardinality}}, edge=violet, edge label={node[midway,left]{\code{left}}}]
    [   {\color{violet}CallExpression \code{call}}, edge=violet, edge label={node[midway,right]{\code{right}}}]]
    [  {\color{teal}CallExpression \code{call}}\\{\color{magenta}Expression \code{returnValue}}, edge label={node[midway,right]{\code{right}}}]]]
  \end{forest}
  \caption{
    The AST of a Mock-, Stub-, or CombinedInteractionDeclaration,
    matching colors with their corresponding source code structures
    in Figure~\ref{fig:InteractionBlockVariants}.
    The \textit{calls} are collapsed to save space;
    their homogeneous subtrees are shown in detail in Figure~\ref{fig:InteractionBlockCallAst}.
  }\label{fig:InteractionBlockAst}
\end{figure}

\begin{figure}
  \centering \setminted{fontsize=\footnotesize}
  \begin{forest}
  for tree={l=1.6cm}
    [{CallExpression \code{call}}
    [ {Expression \code{mockObject}}, edge label={node[midway,left]{\code{callee}}}]
    [ {Expression \code{args}}, edge label={node[midway,right]{\code{arguments}}}]]
  \end{forest}
  \caption{
    The AST of a \textit{call} as it appears in multiple places
    in Figure~\ref{fig:InteractionBlockAst}.
  }\label{fig:InteractionBlockCallAst}
\end{figure}
