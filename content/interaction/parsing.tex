\section{Interaction declaration parsing}
The first step of any \textit{declare} transformation
is parsing the interaction declaration.
The three different interaction declaration types defined by the syntax
are represented as a tagged union \code{InteractionDeclaration}
with the discriminator property \code{kind}
as shown in lines 7 and 12 and 17
of Figure~\ref{fig:InteractionDeclarationType}.
The parser we will now build has the following function signature:
\begin{minted}{typescript}
(NodePath<BinaryExpression>) => InteractionDeclaration;
\end{minted}
In the following, we will use pseudocode to build the parser top-to-bottom.
The full source code is included in the appendix.

\begin{figure}
  \begin{minted}[linenos,fontsize=\scriptsize]{typescript}
export interface BaseInteractionDeclaration {
  mockObject: BabelTypes.Expression;
  args: (BabelTypes.Expression | BabelTypes.SpreadElement)[];
}
export interface MockInteractionDeclaration
  extends BaseInteractionDeclaration {
  kind: 'mock';
  cardinality: BabelTypes.Expression;
}
export interface StubInteractionDeclaration
  extends BaseInteractionDeclaration {
  kind: 'stub';
  returnValue: BabelTypes.Expression;
}
export interface CombinedInteractionDeclaration
  extends BaseInteractionDeclaration {
  kind: 'combined';
  cardinality: BabelTypes.Expression;
  returnValue: BabelTypes.Expression;
}
export type InteractionDeclaration =
  | MockInteractionDeclaration
  | StubInteractionDeclaration
  | CombinedInteractionDeclaration;
  \end{minted}
  \caption{
    The tagged union type \code{InteractionDeclaration}.
  }\label{fig:InteractionDeclarationType}
\end{figure}

The entry point function \code{parseInteractionDeclaration}
looks at the operator used in the top-level binary expression
and decides whether the input is
a mock interaction declaration with operator \code{'*'} (line 3)
or a stub or combined interaction declaration with operator \code{'>>'} (line 5):
\begin{minted}[linenos,fontsize=\footnotesize]{text}
parseInteractionDeclaration(expressionPath):
  switch (expressionPath.node.operator)
    case '*'
      return parseMockInteractionDeclaration(expressionPath)
    case '>>'
      return parseStubInteractionDeclaration(expressionPath)
    default
      throw 'unexpected operator'
\end{minted}

The sub-parser function \code{parseMockInteractionDeclaration} as shown below simply
uses the left side of the binary expression as the \textit{cardinality}
in line 5
and parses the right side of the binary expression as a \textit{call}
in line 3:
\begin{minted}[linenos,fontsize=\footnotesize]{text}
parseMockInteractionDeclaration(expressionPath):
  return
    mockObject, args = parseCall(expressionPath.get('right'))
    kind = 'mock'
    cardinality = expressionPath.node.left
\end{minted}

The sub-parser function \code{parseStubInteractionDeclaration} as shown below
behaves similarly for a pure stub interaction declaration,
using the right side of the binary expression as the \textit{returnValue} in line 10 or 15
and parsing the left side of the binary expression as a \textit{call} in line 13.
However, if the left side is a binary expression with operator \code{'*'},
there is a combined interaction declaration on hand,
and the left side is instead parsed as a mock interaction declaration in lines 7 to 8:
\begin{minted}[linenos,fontsize=\footnotesize]{text}
parseStubInteractionDeclaration(expressionPath):
  let leftPath = expressionPath.get('left')
  let returnValue = expressionPath.node.right
  if (leftPath.isBinaryExpression())
    if (leftPath.node.operator == '*')
      return
        mockObject, args, cardinality =
          parseMockInteractionDeclaration(leftPath)
        kind = 'combined'
        returnValue = returnValue
    throw 'unexpected operator'
  return
    mockObject, args = parseCall(leftPath)
    kind = 'stub'
    returnValue = returnValue
\end{minted}

The sub-parser function \code{parseCall} as shown below
simply extracts the \textit{callee} as the \textit{mockObject} in line 4
and the \textit{arguments} as the \textit{args} in line 5
from a \textit{CallExpression}:
\begin{minted}[linenos,fontsize=\footnotesize]{text}
parseCall(expressionPath):
  if (expressionPath.isCallExpression())
    return
      mockObject = expressionPath.node.callee
      args = expressionPath.node.arguments
  throw 'unexpected expression type'
\end{minted}
