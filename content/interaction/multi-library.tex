\section{Multi-library support}
Unlike the common operation of parsing interaction declarations,
generating statements that perform the declaration after parsing
as well as generating statements that perform the verification
are more complex tasks that require proper abstractions in order
to work with multiple supported mocking libraries.

\subsection{Purpose}
There are multiple reasons why we want to provide support
for different mocking libraries as opposed to
supporting just a single one or even shipping our own.

\paragraph{Distinctive library features}
Users may wish to use special functionality
provided by their mocking library of choice
alongside the usage through spockjs.
This is particularly common for libraries
that are included in a test runner and
designed to integrate well with it.

For example, Jest \autocite{JestGithub}
allows replacing the module \code{module.js}
with a mock implementation \code{__mocks__/module.js}
wherever the module is used by simply
calling \code{jest.mock('./module.js')} in a test.
\autocite{JestMockFunctions}

\paragraph{Avoiding technology lock-in}
Forcing the usage of a specific mocking library,
possibly even one that is only used for spockjs,
would also significantly increase the severity of
technology lock-in imposed by spockjs
by preventing users from returning to
direct usage of the mocking library
if they wish to stop using spockjs.

Indeed, avoiding this has been a focus
for the implementation of assertion blocks.
The spockjs documentation has a section
dedicated to an automated escape hatch
that allows users to transform \textit{(codemod)}
all of their tests to make conventional assertion calls,
allowing them to drop spockjs without first
undergoing an enormous refactoring effort.
\autocite{SpockjsGithub}
