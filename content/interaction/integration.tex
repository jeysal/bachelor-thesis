\section{Integration}
The parser is part of the new \textit{@spockjs/interaction-block} package,
but it is not the only component of the package,
let alone the top-level \textit{index.ts} file.
Instead, \textit{@spockjs/interaction-block} exports a function with the signature:
\begin{minted}[fontsize=\scriptsize]{typescript}
(t, config) => ({
  declare: (statementPath: NodePath<BabelTypes.Statement>) => void,
  verify: (statementPath: NodePath<BabelTypes.Statement>) => void,
});
\end{minted}

This signature is reminiscent of \textit{@spockjs/assertion-block},
because it, too, performs the task of transforming a declarative block into
something that actually executes meaningful code.
However, this package can transform two distinct types of blocks
--- interaction declaration and interaction verification blocks ---
so after applying the \code{t} and \code{config} arguments
that do not change throughout the compilation,
there are both the \textit{declare} and the \textit{verify} function available
to be applied to individual blocks with the
respective labels \textit{mock/stub} and \textit{verify},
which are also exported by the package as
\textit{declarationLabels} and \textit{verificationLabel}.

Early during the \textit{declare} transformation,
the parser we implemented previously will always be called on the \textit{statementPath}.
Afterwards, the transformation must generate calls to the mocking library in use.
We will see that there are multiple ways to implement the rest of the transformation
while supporting multiple mocking libraries.
We will get to those later in this chapter and in the following chapters.

To integrate the new interaction block transformation package into the
central \textit{@spockjs/babel-plugin-spock} Babel plugin package
--- similarly to the assertion block transformation package ---
we first import all of its exported values:
\begin{minted}[fontsize=\scriptsize]{typescript}
import transformInteractionDeclarationStatement, {
  declarationLabels as interactionDeclarationLabels,
  verificationLabel as interactionVerificationLabel,
} from '@spockjs/interaction-block';
\end{minted}
Then we add if statements that handle the new block types
right below the assertion block handling
(after line 51 in the plugin source code that was
previously shown in Figure~\ref{fig:BabelPluginSpockBefore}):
\begin{minted}[fontsize=\scriptsize]{typescript}
// interaction block
else if (interactionDeclarationLabels.includes(label)) {
  transformLabeledBlockOrSingle(
    transformInteractionDeclarationStatement(babel.types, config).declare,
    path,
  );
} else if (label === interactionVerificationLabel) {
  transformLabeledBlockOrSingle(
    transformInteractionDeclarationStatement(babel.types, config).verify,
    path,
  );
}
\end{minted}
