\section{Mocks}
There is no clear consensus on the meaning and differentiation
of the terms `stub' and `mock', however, a common definition
--- and the one that we will use ---
is that mocks extend stubs to
not only answer to calls with predefined responses,
but also to place equally predefinable assertions on
the calls that the test subjects performs on the mock,
\autocite{MocksArentStubs}
thus becoming part of the test oracle.
Figure~\ref{fig:Mock} shows the same example as for stubs before,
but using a mock that introduces an assertion
on the behavior of the test subject.
Many mocking libraries choose to
not make a distinction at all in practice
and have their mock objects act as
dummies, fakes, stubs, spies and mocks
--- however those may be distinguished --- at once.
\autocite{SpockFrameworkDoc}\autocite{JestMockFunctions}

\begin{figure}
  \begin{minted}[linenos]{javascript}
const subtract = add => (a, b) => add(a, -b);

const addCalls = [];
const addMock = (a, b) => (
  addCalls.push({ a, b }), a === 3 && b === -2 ? 1 : 0
);

assert(subtract(addMock)(3, 2) === 1);
assert.deepStrictEqual(addCalls, [{ a: 3, b: -2 }]);
  \end{minted}
  \caption{
    Using a mock defined in lines 4 to 6
    to verify in line 9
    that the \code{subtract} function defined in line 1
    actually made a call to \code{add}.
    Note that this concrete example would typically be considered
    a significant overspecification in practice.
  }\label{fig:Mock}
\end{figure}

Figure~\ref{fig:MockLibrary} shows how a mocking library
can greatly simplify the mocking example
by eliminating the manual argument capturing from the test code
and providing call verification helper functions.

\begin{figure}
  \begin{minted}[linenos]{javascript}
const subtract = add => (a, b) => add(a, -b);

const addMock = jest.fn(
  (a, b) => (a === 3 && b === -2 ? 1 : 0)
);

expect(subtract(addMock)(3, 2)).toBe(1);
expect(addMock).toBeCalledWith(3, -2);
  \end{minted}
  \caption{
    Getting rid of mocking boilerplate by using a mocking library,
    in this example the one provided by the test runner \textit{Jest}.
    \autocite{JestGithub}
    Note that unlike the previous `vanilla' example,
    this will allow additional calls to the mock
    as long as the first one is performed with matching arguments
    because of the way Jest implements the call matching.
  }\label{fig:MockLibrary}
\end{figure}
