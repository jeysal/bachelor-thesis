\section{Test stubs}
\textit{Test stubs} are the inverse counterpart to test drivers.
They replace components that the test subject depends on
and provide canned answers to calls from the test subject to the stubbed component
\autocite{MocksArentStubs}
as shown in Figure~\ref{fig:TestStub}.
They are particularly useful in top-down software development,
where the high-level components are implemented first
and testing them requires stubs for their dependencies
that are not yet implemented in full detail.

\begin{figure}
  \begin{minted}[linenos]{javascript}
const subtract = add => (a, b) => add(a, -b);

const addStub = (a, b) => (a === 3 && b === -2 ? 1 : 0);

assert(subtract(addStub)(3, 2) === 1);
  \end{minted}
  \caption{
    In this somewhat contrived stubbing example,
    the \code{add} function,
    a dependency of the \code{subtract} function defined in line 1,
    is being stubbed with the \code{addStub} function defined in line 3,
    which does not actually perform any addition logic,
    but instead only responds to a certain call with a canned answer.
    This way, \code{subtract} can be tested in isolation in line 5,
    without an actual implementation of \code{add}.
  }\label{fig:TestStub}
\end{figure}
