\section{Test drivers}
\textit{Test drivers} control the test execution from the outside
\autocite{TestingTheoryDriver}
as shown in Figure~\ref{fig:TestDriver}.
They can replace high-level system components that will
at some point integrate the \textit{test subject},
but have not yet been implemented,
and thus they are particularly useful in conjunction with
a bottom-up software development approach,
where small components at the bottom of the hierarchy
are first designed in their entirety
and later integrated to form the software product as a whole.

Test drivers need not replace a future system component.
They can also be used in place of components that
have already been implemented,
if it is still desired that the components it depends on
should be tested on their own.
They can also be used in place of components that
will never be implemented
because they are out of scope for the system being developed,
in particular if the test drivers replace the human users who interact
with the top-level components of the system in production themselves.
\autocite{ArtOfSoftwareTesting}

\begin{figure}
  \begin{minted}[linenos]{javascript}
import test from "ava";

import add from "./add";

test("adds two positive integers", t => {
  t.is(add(1, 2), 3);
});

test("adds a positive and a negative integer", t => {
  t.is(add(1, -2), -1);
});
  \end{minted}
  \caption{
    An example test driver that uses the testing framework AVA
    \autocite{AvaGithub}
    to execute two test cases for the test subject \code{add}.
  }\label{fig:TestDriver}
\end{figure}
