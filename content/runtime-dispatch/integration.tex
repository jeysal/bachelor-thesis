\section{Interaction runtime adapter integration}
With the runtime dispatch approach,
the \textit{@spockjs/interaction-block} package
--- apart from the parser ---
looks very different and has to do a lot more work.

One part that looks somewhat similar is
the outer function:
\begin{minted}[linenos,fontsize=\scriptsize]{typescript}
export default (t: typeof BabelTypes, config: InternalConfig) => {
  const {
    hooks: { interactionRuntimeAdapter, interactionVerificationPostProcessors },
  } = config;
  const postProcessors = interactionVerificationPostProcessors.map(processor =>
    processor(t, config),
  );

  if (!interactionRuntimeAdapter) {
    throw new Error(/* ... */);
  }

  return {
    declare, // ...
    verify, // ...
  };
};
\end{minted}
Instead of ensuring there is at least one primary interaction processor,
we now ensure that there is an interaction runtime adapter defined.

One might notice that there are still some sort of interaction processors
in this implementation, called \textit{interactionVerificationPostProcessors}.
These are still required as a kind of replacement for
the non-primary interaction processors from the direct compilation approach,
because for example \textit{@spockjs/preset-runner-ava} needs to
add a passing assertion after the mock verification
because AVA makes tests fail if they run without performing an assertion through AVA.

\subsection{Declaration serialization}
The \textit{declare} function does quite a lot more
than just delegating through with the runtime dispatch approach.
After the usual AST structure checks and the call to the parser,
we need to \textit{serialize} the compile-time interaction declaration
to AST nodes that will create a runtime interaction declaration object
to be passed to the interaction runtime adapter when the code is executed.
Generating the adapter call is simple:
\begin{minted}[linenos,fontsize=\footnotesize]{typescript}
statementPath.replaceWith(
  t.callExpression(
    addNamed(
      statementPath,
      'declare',
      interactionRuntimeAdapter,
    ) as BabelTypes.Expression,
    [t.objectExpression(interactionDeclarationProperties)],
  ),
);
\end{minted}

Babel provides helper functions to conveniently create imports
in its \textit{babel-helper-module-imports} package.
We use the \textit{addNamed} function to generate an import like
\begin{minted}[fontsize=\footnotesize]{javascript}
import {
  declare,
} from '@spockjs/interaction-runtime-adapter-example';
\end{minted}
in lines 3 to 7.
Its return value is an expression evaluating to the imported value,
which is the \textit{declare} function of the interaction runtime adapter,
so we can generate a call to that function
with a runtime interaction declaration as the argument in line 8.

Obtaining the \textit{interactionDeclarationProperties} requires some more work.
The first two properties are simple ---
the \textit{kind} and \textit{args} property nodes
are created according to the following pattern:
\begin{minted}{typescript}
t.objectProperty(
  t.identifier('kind'),
  t.stringLiteral(interactionDeclaration.kind),
)
\end{minted}
A \textit{cardinality} property is added in the same way,
but only if the interaction declaration
is of type \textit{mock} or \textit{combined}.
A \textit{returnValue} property is added in the same way,
but only if the interaction declaration
is of type \textit{stub} or \textit{combined}.

Finally, the \textit{mockObject} property is a bit more complex.
When implementing the first approach,
we have already noticed that for the Sinon.JS mocking library
we need to distinguish between member expressions
and other expressions in the mock object,
because the former must result in an \textit{expects} call to Sinon.JS.
This distinction cannot be deferred to the runtime,
because there is no way for the runtime adapter
to find out how the \textit{mockObject} value
passed in the runtime interaction declaration was obtained.

So we introduce another property
on the \textit{RuntimeInteractionDeclaration} types:
The \textit{methodName}.
If the mock object was specified as a member expression,
the \textit{mockObject} property will only contain
the object that the member access occurs on,
while the \textit{methodName} property specifies
the name of the member of that object.
Otherwise, \textit{methodName} will be \code{undefined}.
This distinction is implemented mostly in the same way
as in the direct compilation approach;
the full source code of \textit{@spockjs/interaction-block}
can be found in the appendix.

\subsection{Verification}
TODO
