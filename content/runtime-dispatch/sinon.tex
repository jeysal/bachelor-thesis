\section{Sinon.JS}
To implement \textit{@spockjs/interaction-runtime-adapter-sinon},
we use the same Sinon.JS (\textit{Sinon}) APIs
as with the direct compilation approach,
but because we are now operating at runtime,
we can call them directly instead of generating code that does so.
We can also simply branch to handle
different kinds of interaction declarations and mock objects
instead of building templates for all combinations.

The following code implements the \textit{declare} function of the adapter:
\begin{minted}[linenos,fontsize=\scriptsize]{typescript}
declaration => {
  let mock = declaration.mockObject;

  const { methodName } = declaration;
  if (methodName != null) {
    mock = mock.expects(methodName);
  }
  mock = mock.withArgs(...declaration.args);

  if (declaration.kind === 'mock' || declaration.kind === 'combined') {
    const { cardinality } = declaration;
    mock = mock.atLeast(cardinality).atMost(cardinality);
  }

  if (declaration.kind === 'stub' || declaration.kind === 'combined') {
    mock = mock.returns(declaration.returnValue);
  }
};
\end{minted}
In lines 4 to 7, we handle mock objects with methods in Sinon
by calling the \textit{expects} function if
the \textit{methodName} declaration property
that we introduced for this special case is set.
In line 8, we restrict the call matching
to calls with the correct arguments as specified in the declaration.
In lines 10 to 17,
depending on the declaration kind,
we tell Sinon to expect a certain number of matching calls,
or to stub the return values on call, or to do both.
The \textit{verify} function is as simple as:
\begin{minted}[linenos,fontsize=\scriptsize]{typescript}
mock => {
  mock.verify();
};
\end{minted}

Additionally, we could introduce checks in the future to ensure
that the given mock objects are really Sinon mocks
--- especially now that performing additional steps at runtime
does not require adding code to a template to
insert them at every interaction block ---
and throw an error with a helpful message if this is not the case
instead of the obscure type errors that the adapter might run into
if a user mistakenly attempts to declare or verify interactions
on a value that is not a Sinon mock.
