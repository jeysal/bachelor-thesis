\section{Jest}
To implement \textit{@spockjs/interaction-runtime-adapter-jest},
we use the same custom interaction declaration management
on top of the Jest mock primitive
that we already developed with the direct compilation approach.
We will not repeat the code here,
as it is very similar to the template contents.
The full source code of this implementation,
along with test cases for both implementation approaches
of Jest-backed spockjs interaction blocks,
is included in the appendix.

One small difference is that the \textit{isMockFunction} check
is converted from a negative condition with early return error handling
to a positive condition with else block error handling.
Because unlike the template literal, the adapter code can be type-checked,
this allows \textit{isMockFunction} to act as a \textit{type guard}
so the TypeScript type checker knows that
the value must be a Jest mock inside the if block.

We also need to incorporate some extra logic
compared to the direct compilation approach,
because the generic interaction block handler
always rewrites declarations on a member expression
with the \textit{methodName} property,
not only when Sinon is used.
Even though Jest does not need this special case,
we have to handle it at the start of the \textit{declare} function:
\begin{minted}[linenos]{typescript}
const { args, methodName } = declaration;
let { mockObject } = declaration;

if (methodName != null) {
  mockObject = mockObject[methodName];
}
\end{minted}
Afterwards, the \textit{mockObject} will be
the evaluated result of the entire expression again,
regardless of whether it was disassembled as a member expression,
or directly passed to us as a generic expression.
We continue the rest of the \textit{declare} function,
as well as the \textit{verify} function,
with essentially the same code as in the direct compilation approach.
