\section{Runtime dispatch: The good}
The runtime dispatch approach
provides a significantly better experience
for development of spockjs itself,
making it easier to maintain and scale.
Writing most of the library-specific code
in interaction processor template strings
gets cumbersome rather quickly,
while interaction runtime adapter code
can be written and type-checked
like any other spockjs code,
barely even revealing that
it is executed at test runtime
instead of test compile time.

For mocking libraries like Jest
that do not provide APIs
implementing sufficient interaction matching,
thus requiring us to implement much of it ourselves,
lackluster developer experience (\textit{DX})
can rapidly become a limiting factor to both quantity and quality
of the functionality that we can provide for those libraries.

It was already annoying to implement
the cardinality checks in Jest mock verification.
Imagine implementing and debugging an algorithm
to match \textit{ordered} mock expectations
in the templates of an interaction processor.
Imagine adding improved error messages for
all kinds of mistakes users of spockjs might make
using nested template strings
with escaped characters all over.
In a runtime adapter,
these tasks can be performed
using plain TypeScript code.

It is also easier to write the code \textit{well} this way.
For instance, while both approaches use \textit{Symbols} \autocite{MdnSymbol}
to `hide' interaction declarations inside a Jest mock instance,
the direct compilation approach uses
a named Symbol retrieved by the code
\code{Symbol.for('spockjs...')}
to access the declarations everywhere,
while the runtime adapter approach can just
generate a \code{Symbol()} once and use that
to access the declarations everywhere,
giving it a Symbol that is guaranteed to be unique,
which is arguably a cleaner way that follows best practices.

If this were a requirement to be implemented with the direct compilation approach,
which does not have the ability to easily hold global state
like a runtime adapter does in the top-level scope of its module,
we would need to either generate a Symbol constant
into the top-level scope of each test file
--- which is easily done using Babel,
but ceases to work if a mock object is passed around across test files
because they would each have their own distinct Symbol ---
or we would need to store the Symbol itself
somewhere in the global scope,
which is not really cleaner or safer than using a named symbol
as is being done in the current direct compilation implementation.

To improve upon the general spockjs DX issues of direct compilation,
it might be viable to build advanced infrastructure
that allows us to move template contents into separate files
that can be type-checked and otherwise used regularly in development,
transform those, parse them as templates and perform the substitutions.
The viability of such intricate infrastructure extensions
is questionable compared to using runtime dispatch right away,
which appears to be the cleaner concept for optimizing the DX.
