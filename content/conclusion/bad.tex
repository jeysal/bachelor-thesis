\section{Runtime dispatch: The bad}
As far as spockjs DX is concerned,
there is also one negative aspect to be remarked about runtime dispatch.
The spockjs integration tests now need yet another transpilation step,
resulting in even more complexity in the development setup.
In addition to
\begin{itemize}
  \item transforming the spockjs test files with spockjs itself
    (which is necessary because spockjs is \textit{self-hosting}
    in that its very own tests use spockjs blocks as well),
  \item transforming the spockjs TypeScript source files
    in order to perform aforesaid transformation, and
  \item transforming test files in the spawned test runner child process
    (because \textit{integration} tests always perform
    a full test runner execution like a user would),
\end{itemize}
we now also need the spawned child process
to be capable of compiling TypeScript on-the-fly
in the runtime adapter modules imported by the test files.

As confusing as this may sound,
it is somewhat specific to the spockjs development infrastructure,
it is not an unsolvable problem in development, and
it does not arise at all in production,
where the spockjs TypeScript source code is transpiled ahead-of-time.
To see how the the various transpilation steps
used during spockjs development are set up,
refer to the top-level and integration test directories of spockjs.
\autocite{SpockjsGithubRuntimeDispatch}

A more significant downside to the runtime adapter approach is
that the automated escape hatch
for users who want to stop using spockjs
mentioned in Chapter~\ref{chap:Interaction}
can no longer be used to obtain test files
that work completely independent of spockjs.
With the direct compilation approach,
at least for test runners like Sinon,
the escape hatch will just
replace the interaction blocks
with the corresponding Sinon API calls.
Even for Jest, it will produce working tests
by inserting all of the interaction matching logic all over the tests,
although that of course does not
leave the tests in a maintainable state.
With the runtime adapter approach,
the escape hatch will leave the tests
with imports and calls to
a spockjs interaction runtime adapter in them,
which some may still consider valuable
because the tests no longer need
a transformation step for spockjs blocks,
but it does not fully meet
the purpose of the escape hatch,
that is removing all traces of spockjs
as if it was never used.

While the runtime dispatch approach is arguably
more comfortable for spockjs development,
it lags behind a bit in
achieving great user experience (\textit{UX}).
Showing a snippet of the original code in an error message,
like the direct compilation Jest interaction processor does,
is not possible in a runtime adapter
unless we serialize some more information from the AST
in the library-agnostic interaction block handling routine.
This leads us to the more general problem space
of information loss during transition from compile time to runtime.
