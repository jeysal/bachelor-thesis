\section{Information loss}
During compilation of tests with the spockjs Babel plugin,
we naturally incur some information loss.
At runtime, we will no longer have access to the AST data.
If we need to access that data,
for example to print better error messages,
we have to explicitly preserve the information during compilation,
which can be done to different degrees, such as
\begin{itemize}
  \item serializing pieces of information
    as they are needed at runtime, e.g.
    \code{"mockName": "(0 || mock)"},
  \item serializing the \textit{mockObject} AST node,
    which can be printed or used in other ways at runtime, and
  \item serializing the entire interaction declaration subtree
    for maximum information preservance,
    which is usually overkill and can have
    a noticeable performance impact.
\end{itemize}

We also need to be wary about self-inflicted information loss.
Suppose that Sinon had no
\textit{atLeast} and \textit{atMost} functions
for specifying cardinality \textit{n},
instead requiring users to
register the same mock expectation \textit{n} times.
The correct handling of this situation would be
to loop \textit{n} times at runtime
to register those expectations with Sinon.
If we instead decided to unfold the declarations at compile time so that
\code{2 * fn()} becomes the same as \code{1 * fn(); 1 * fn()}
in our internal declaration data structure,
we could no longer use any opportunity
to specify the cardinality directly
offered by other mocking libraries,
and we would be forced to always use complex queueing declaration semantics
as described in Chapter~\ref{chap:Interaction},
including for our own declaration matching implementations
such as the Jest adapter.
Deferring the computation of derived data representations to runtime
can save us from a lot of trouble.

The kind of self-inflicted information loss we described here is always avoidable;
it is just a matter of designing the right abstractions and data structures.
The natural information loss during compilation
is avoided entirely with the direct compilation approach
by doing all the work at compile time straight away;
if we urgently need the naturally lost information,
the runtime dispatch approach requires
extra serialization effort for all of this information.
As long as we do not need large parts of this information though,
runtime dispatch is a lot easier to handle for us developers.

TODO hybrids possible, processors that hook into *how* a runtime adapter call is generated at runtime, serializing additional compile-time info and doing stuff with it directly or passing it to the adapter
TODO moreover, mixing possible:
TODO    make the runtime adapter approach impl of interaction-block just another "processor" in the direct compilation approach.
TODO    -> Sinon can just specify its interaction-processor-sinon-mocks directly
TODO    -> Jest can specify the serializing, to-runtime-delegating processor and pass its interaction-runtime-adapter-jest as an argument to the processor
TODO    this way, runtime dispatch is kind of embedded into direct compilation

TODO perhaps connect this to how projects should decide what to do at compile time / runtime based on what information they need?
