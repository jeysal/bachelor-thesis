\section{Architecture}
Although the entire codebase of spockjs resides in
a single source code repository (\textit{monorepo}),
it is organized in and distributed as various packages
on the \textit{npm (Node.js package manager)} registry,
\autocite{Npm}
the de facto standard place for publishing JavaScript code.
The packages are all published in the \textit{@spockjs} namespace.
We have already seen two of these packages:
\textit{@spockjs/babel-plugin-spock} and \textit{@spockjs/assertion-block}.
There are a total of nine packages,
and we will add some more when implementing interaction testing.

Inside the repository, a package \textit{@spockjs/name}
is located in a \textit{packages/name} directory.
The `workspaces' feature \autocite{YarnWorkspacesDoc}
of the package manager \textit{Yarn}
links them all together during development
so changes across multiple packages do not require
first publishing the changes to a dependency
before working on the changes to a dependent.
In addition, the tool \textit{Lerna} \autocite{LernaGithub} is used to
perform tasks such as builds, changelog generation, version bumps, and releases
on multiple packages at once.
