\section{Babel}
Babel is a popular community-developed JavaScript compiler
that is used and sponsored by many of the most prominent tech companies
and has acquired more than 28000 stars on GitHub as of mid-July 2018.
\autocite{BabelGithub}
Started as `6to5', serving the purpose of transforming
ECMAScript 2015 (at that time named ECMAScript 6) code to
ECMAScript 5 code that would run in all browsers and on all platforms
regardless of their ECMAScript 2015 support,
Babel can now not only transform input code using
more recent versions of ECMAScript such as ECMAScript 2018
and even ES.Next proposals that are not yet in the standard,
but also apply arbitrary plugins developed by the community
that can transform the code to their liking.
\autocite{BabelPluginPrevalGithub}
\autocite{PowerAssertGithub}

In the following, we will acquire some basic knowledge on how to develop Babel plugins.
For a more extensive and detailed view that would go beyond the scope of this paper,
both the author and the official Babel documentation recommend the Babel Plugin Handbook.
\autocite{BabelPluginHandbook}

\subsection{Plugin visitors}.
Babel plugins apply AST transformations in \textit{visitors}
as introduced in Chapter~\ref{chap:Ast}.
A Babel plugin is essentially no more than a module that exports
a function returning an object with arbitrary visitor functions.
The following is the full Babel plugin to accomplish
the strict equality swap example transformation
as described in Chapter~\ref{chap:Ast}:
\begin{minted}[linenos]{javascript}
export default ({ types: t }) => ({
  visitor: {
    BinaryExpression({ node }) {
      if (node.operator === "===") {
        [node.left, node.right] = [node.right, node.left];
      }
    }
  }
});
\end{minted}

The \code{types} property of the object passed to the plugin
in line 1 is not used in this particular plugin,
but is usually one of the most important helpers,
providing creation methods for each type of node
and the corresponding type check methods:
\begin{minted}{javascript}
t.isIdentifier(t.identifier('something')) === true
\end{minted}

The visitor object returned from the plugin
can contain a visitor method for each node type (capitalized),
but it may be partial, i.e.
it does not have to implement methods for all of the more than 200 node types.
In this case, we only provide a visitor for binary expressions in lines 3 to 7.
The visitor method receives the current \code{path} object as its first parameter,
which we will later take a closer look at,
but in this case,
we extract only the binary expression node from it in line 3.
There is also a second \code{state} parameter
that we can disregard here, but it is for example needed
to access user-supplied configuration options that a plugin might offer.

The remaining part of the plugin inside the binary expression visitor
is the code developed in Chapter~\ref{chap:Ast}
and performs the transformation as already explained there.

\subsection{Nodes vs.\ paths}
Visitor methods receive \textit{path} objects when they are called while the AST is traversed.
Unlike nodes, which almost exactly reflect the node objects as defined by ESTree,
paths are aware of their context in the tree and, in particular, their parents.
This means that while a path can be used to reach and manipulate any part of the tree,
including those above the corresponding node,
a node can only be used to inspect and manipulate its children.

We have already seen how to navigate and manipulate the primitive node objects,
so let us now take a quick look at the more powerful path objects,
covering only a select few of their many functions that are relevant to spockjs.

To retrieve child paths, \code{get} can be used,
specifying the same property access path
that would be taken on a node as a string argument.
The following example describes the path navigations using \code{get} on the left
by comparing them to their corresponding node navigations on the right:
\begin{minted}{javascript}
path.get('left').node   === path.node.left
path.get('body.0').node === path.node.body[0]
\end{minted}
While there is no reason to use the more cumbersome path navigation in this example,
we need to use it if we want to call methods on the child path,
some of which are named in the following.
There is no way to recover the path object
if we only have its corresponding node object.

Spockjs frequently uses the \code{replace} method,
which discards the entire node that belongs to the path
and replaces it with a given node,
and its sibling method \code{replaceWithMultiple},
which replaces it with all nodes in a given array,
provided that the surrounding context allows that to happen.

In fatal error cases, the utility method \code{buildCodeFrameError}
is highly useful to bail with an error message
that tells the user where in the source code
the bit that the plugin doesn't understand is located.
