\section{Installation}
As noted earlier, spockjs is compatible with most modern JavaScript test runners
--- it is \textit{not} a test runner by itself,
which sets it apart from the Spock Framework.
Instead, spockjs ships in the form of a plugin for the popular JavaScript compiler Babel,
\autocite{Babel}
which can be set up to work with most modern test runners.
\autocite{BabelSetupDoc}
We will look at how spockjs uses Babel to implement assertion blocks later in the chapter.

\subsection{Configuration}
Spockjs is designed to work well by default,
but does offer a few configuration options,
mostly to turn miscellaneous features on or off.
\autocite{SpockjsGithub}

\paragraph{powerAssert}
This on-by-default option causes spockjs to generate
detailed error messages when an assertion fails
using an external library provided by the \textit{power-assert} project.
\autocite{PowerAssertGithub}
The following is an example of how such an error message can look:
\autocite{SpockjsGithub}
\begin{minted}{text}
assert(3 * 3 >= 4 * 4)
         |   |    |
         |   |    16
         9   false
\end{minted}

\paragraph{autoImport}
This on-by-default option causes spockjs to
automatically import necessary modules
--- such as the \code{assert} core module ---
into the test file.
Without this feature, users would need to
manually add an import to each test file,
which is easy to forget and might cause
other tools to warn about the import being supposedly unused.

\paragraph{staticTruthCheck}
This off-by-default option causes spockjs to
use the static analysis capabilities of Babel
in order to detect assertions
that can be inferred to always be truthy or always be falsy,
which indicates that the assertion might not provide any value to the test case,
and throw an error without even executing the test.
The following is an example of such a useless assertion:
\begin{minted}{javascript}
const x = 1;
expect: x === 1;
\end{minted}

\paragraph{assertFunctionName}
This string option can override the name of the assert function spockjs uses,
which can be useful if \textit{autoImport} is not used,
or for the \textit{escape route} that spockjs provides
for users who want to stop using spockjs without rewriting all their tests,
which we will get back to in a later chapter.
By default, the name of the assert function is \textit{'assert'}
or whichever free identifier is chosen for the automatic import.

\paragraph{presets}
This array of strings can be used to apply so-called \textit{presets},
which can hook into the transformation logic of spockjs
and customize its behavior in various ways.
We will analyze the preset mechanism in greater detail
when talking about the architecture of spockjs.
Spockjs currently provides two `official' presets:
\begin{itemize}
  \item \textit{@spockjs/runner-jest} ensures helpful assertion error messages
    when using the test runner Jest, \autocite{JestGithub}
    which would otherwise hide too many details
    due to its special handling of assertion errors.
  \item \textit{@spockjs/runner-ava} eliminates the necessity to
    set an AVA \autocite{AvaGithub} configuration option
    that makes test cases fail erroneously if not set,
    and it causes assertion error messages to look
    exactly the same as if native AVA assertions were used.
\end{itemize}
