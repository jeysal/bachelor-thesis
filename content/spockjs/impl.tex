\section{Implementation}
In this section, we examine the existing implementation of assertion blocks in spockjs, broken down to
\begin{enumerate}
  \item the detection and processing of blocks in general, and
  \item assertion transformation specifically.
\end{enumerate}

\subsection{Blocks}
The \textit{@spockjs/babel-plugin-spock} package is the entry point that exposes a Babel plugin ---
we will talk about the multi-package structure of spockjs later in the chapter.
The full source code of this package is shown in Figure~\ref{fig:BabelPluginSpockBefore}.

The Babel plugin in lines 39 to 54 consists of just a single visitor method for labeled statements.
The visitor checks whether the label name of the current statements
is one of the known labels that designate an assertion block
--- \mintinline{typescript}{['expect', 'then']} --- and if so,
it calls the \textit{transformLabeledBlockOrSingle} helper defined in lines 11 to 37
to apply the \textit{assertifyStatement} transformation
to each statement in the case of a block statement contained inside the labeled statement,
or otherwise to the single statement contained inside the labeled statement.

Note that the \textit{assertifyStatement} transformation function,
which comes from the \textit{@spockjs/assertion-block} package that we will address after the general block handling,
is a higher-order function with the signature
\begin{minted}{typescript}
(babel, state, config) => (
  statementPath: NodePath<BabelTypes.Statement>
) => void;
\end{minted}
, so after applying the first three arguments,
it can be used any number of times on different statement paths.

The \textit{transformLabeledBlockOrSingle} helper does exactly that for us,
applying the transformation to the statements in a block statement in line 26,
or to a single statement in line 32.
It also removes the labels by
replacing the labeled statement with its body
in line 29 or 35, respectively,
in order to produce cleaner code.
We will be able to reuse this helper when implementing
interaction blocks later on.

\begin{figure}
  \begin{minted}[linenos,fontsize=\scriptsize]{typescript}
import { PluginObj } from '@babel/core';
import { NodePath } from '@babel/traverse';
import * as BabelTypes from '@babel/types';

import { extractConfigFromState } from '@spockjs/config';

import assertifyStatement, {
  labels as assertionBlockLabels,
} from '@spockjs/assertion-block';

const transformLabeledBlockOrSingle = (
  transform: (statementPath: NodePath<BabelTypes.Statement>) => void,
  path: NodePath<BabelTypes.LabeledStatement>,
): void => {
  const labeledBodyPath = path.get('body') as NodePath<BabelTypes.Statement>;

  switch (labeledBodyPath.type) {
    case 'BlockStatement':
      // power-assert may add statements in betweeen,
      // so never reuse body array
      const statementPaths = () =>
        (labeledBodyPath.get('body') as any) as NodePath<
          BabelTypes.Statement
        >[];

      statementPaths().forEach(transform);

      // remove label
      path.replaceWithMultiple(statementPaths().map(stmtPath => stmtPath.node));
      break;
    default:
      transform(labeledBodyPath);

      // remove label
      path.replaceWith(labeledBodyPath);
  }
};

export default (babel: { types: typeof BabelTypes }): PluginObj => ({
  visitor: {
    LabeledStatement(path, state) {
      const config = extractConfigFromState(state);
      const label = path.node.label.name;

      // assertion block
      if (assertionBlockLabels.includes(label)) {
        transformLabeledBlockOrSingle(
          assertifyStatement(babel, state, config),
          path,
        );
      }
    },
  },
});
  \end{minted}
  \caption{
    The source code of \textit{@spockjs/babel-plugin-spock}
    before any interaction testing capabilities were added.
  }\label{fig:BabelPluginSpockBefore}
\end{figure}


\subsection{Assertions}
The \textit{@spockjs/assertion-block} exports the aforesaid function
that can be used to \textit{assertify} a statement,
converting it from an unused expression statement to a function call and
passing the result of evaluating the expression to the assert function.
Figure~\ref{fig:AssertionBlockCore} shows the core part of the package
that performs this transformation,
while omitting some miscellaneous code that is
not essential to the assertion block concept.

\begin{figure}
  \begin{minted}[linenos,fontsize=\scriptsize]{typescript}
(statementPath: NodePath<BabelTypes.Statement>) => {
  if (statementPath.isExpressionStatement()) {
    const statement = statementPath.node;
    const origExpr = statement.expression;

    const assertIdentifier = t.identifier('assert');
    const newExpr = t.callExpression(assertIdentifier, [origExpr]);

    statement.expression = newExpr;
  } else {
    throw statementPath.buildCodeFrameError(
      `Expected an expression statement, but got a statement of type ${
        statementPath.type
      }`,
    );
  }
};
  \end{minted}
  \caption{
    A stripped-down version of \textit{@spockjs/assertion-block}
    without powerAssert, autoImport, staticTruthCheck, assertFunctionName,
    preset post processing, line/column number handling, and
    the enclosing function with the babel, state and config parameters.
  }\label{fig:AssertionBlockCore}
\end{figure}

First of all, if the statement passed to us is not an \textit{ExpressionStatement},
we will not be able to use it as an argument to an assert function call,
so we bail with a compile-time error in lines 11 to 15.
Otherwise, we generate a \textit{CallExpression}
with the \textit{Identifier} `assert' as its \textit{callee},
and the original expression as its single argument in lines 6 to 7
and update the statement to our new expression in line 9.

Figure~\ref{fig:AssertionBlockAst} shows the AST after such a transformation,
with the nodes and edges that have been newly inserted highlighted in teal,
for the example input code
\begin{minted}{javascript}
expect: true;
\end{minted}
that is transformed to:
\begin{minted}{javascript}
expect: assert(true);
\end{minted}

\begin{figure}
  \centering \setminted{fontsize=\scriptsize}
  \begin{forest}
    for tree={l=1.6cm}
    [{Program}
    [ {LabeledStatement}, edge label={node[midway,left]{\code{body[0]}}}
    [  {Identifier (name: \code{expect})}, edge label={node[midway,left]{\code{label}}}]
    [  {ExpressionStatement}, edge label={node[midway,right]{\code{body}}}
    [   {\color{teal}CallExpression}, edge label={node[midway,left]{\code{expression}}}
    [    {\color{teal}Identifier (name: \code{assert})}, edge=teal, edge label={node[midway,left]{\code{callee}}}]
    [    {BooleanLiteral (value: \code{true})}, edge=teal, edge label={node[midway,right]{\code{arguments[0]}}}]]]]]
  \end{forest}
  \caption{An exemplary AST after an assertion block transformation,
    with newly inserted nodes and edges highlighted in teal.}\label{fig:AssertionBlockAst}
\end{figure}
