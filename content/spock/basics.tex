\section{Spock testing DSL basics}
Spock test cases consist of \textit{blocks}.
These blocks are started by a label,
such as \code{setup:}, \code{expect:}, and \code{then:}.
Any code in a test case before a block is declared
implicitly becomes a \textit{setup} block.
\autocite[Chapter: Spock Primer - Feature Methods - Blocks]{SpockFrameworkDoc}
The various types of blocks available
are shown in Figure~\ref{fig:SpockTestExample}.

\textit{Expect} and \textit{then} blocks
can contain assertions formed by simple expressions.
An assertion passes if the expression evaluates to a truthy value
and fails if it evaluates to a falsy value.
Simple functional programming-style assertions
usually look cleaner using the expect style
as shown in lines 2 to 5;
code with side effects should be executed in a \code{when} block,
followed by a \code{then} block verifying that
the effects have resulted in the expected program state
as shown in lines 16 to 26.
It is also possible to assert that an exception
has been thrown in the preceding \code{when} block using \code{thrown},
which can be seen in the \code{when-then} example test case.

The blocks \textit{setup} (with its alias \textit{given}) and \textit{cleanup},
as shown in lines 7 to 14,
are offered to run code before and after the assertions.
\code{cleanup} blocks are particularly useful because they are \textit{always} run,
even if the previously executed blocks threw an exception.
\autocite[Chapter: Spock Primer - Feature Methods - Blocks]{SpockFrameworkDoc}
This functionality makes them useful for tasks
such as restoring the previous state after I/O operations,
as shown in the setup-cleanup example test case.
