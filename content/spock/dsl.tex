\section{Domain-specific languages}
A \textit{domain-specific language (DSL)},
as opposed to a general-purpose language like C or Java,
targets a very specific application purpose.
Examples from various areas of software development are:
\begin{itemize}
  \item \textit{Markdown} for text formatting, \autocite{MarkdownIntro}
  \item the \textit{Gradle Build Language} for build configuration, \autocite{GradleWritingBuildScripts} and
  \item the \textit{Spock specification language} for test cases. \autocite{SpockFrameworkDoc}
\end{itemize}

These three languages can be organized in three rough categories,
distinguished by the way they are implemented:
\begin{enumerate}
  \item Standalone DSLs such as Markdown,
  \item entirely `userspace' DSLs such as the Gradle Build Language, and
  \item as a middle ground, custom compilation DSLs such as the Spock specification language.
\end{enumerate}

\paragraph{Standalone DSLs}
Standalone DSLs are implemented by defining a wholly new
lexical and syntactical grammar and the language semantics.
One can then use a parser generator to obtain a parser for the language
and include in in their software.

The major disadvantages of this approach for a testing DSL are
\begin{itemize}
  \item that the language would have to be enormously complex
    in order to be able to replace the regular Groovy language
    \begin{itemize}
      \item or it would have to be embedded in strings in the test cases, and
    \end{itemize}
  \item that there would be no tooling support for the language
    that provides autocompletion, formatting, error checking, and other assistance
    like there is for widespread existing languages.
\end{itemize}

\paragraph{Userspace DSLs}
\textit{Userspace DSLs} have recently become popular.
They are implemented entirely using features of the host language.
In particular, some languages are explicitly designed to
support the creation of DSLs well, such as
Groovy \autocite{GroovyDslDoc}
and, more recently,
Kotlin \autocite{KotlinTypeSafeBuilderDoc}.
Language features that can aid the creation of DSLs are
\autocite{GroovyDslDoc}
\begin{itemize}
  \item optional parentheses around call arguments,
  \item operator overloading,
  \item closure delegates, and
  \item compilation customization for advanced DSLs of the last category.
\end{itemize}

\paragraph{Custom compilation DSLs}
TODO
