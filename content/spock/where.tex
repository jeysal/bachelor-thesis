\section{Data-driven testing}
Spock also provides a special syntax for \textit{data-driven testing},
shown in the \textit{where} example
in lines 28 to 37 of Figure~\ref{fig:SpockTestExample}.
Using the Data-driven testing features of the Spock Framework,
the same test code can easily be run multiple times,
but passing different inputs to the system under test
and expecting different outputs from it.
\autocite[Chapter: Data Driven Testing]{SpockFrameworkDoc}

The given example declares the test data using a \textit{data table},
resulting in a very compact representation of the data,
separating the data points in the rows
and the variable names in the header
with a single pipe (\code{|}).
A double pipe (\code{||}) is used to separate
inputs (a / b) and expected values (r) clearly;
this difference is only visual,
Spock treats single and double pipes in data tables equally.
\autocite[Chapter: Data Driven Testing - Syntactic Variations]{SpockFrameworkDoc}

\textit{Data pipes} such as \mintinline{groovy}{a << [1, 7, 0]}
can be used as an alternative to data tables and even
allow generating data points dynamically.
However, since data-driven testing is not a primary concern of this paper,
we will not go into more detail for this aspect of the Spock Framework
and instead refer to its documentation.
\autocite{SpockFrameworkDoc}
