\section{Interaction testing}
The Spock Framework comes with an own library
for creating test mocks
(which are at the same time also test stubs).
While it is possible to use any such library
written in Java or another language
that can be compiled to JVM bytecode
in conjunction with the Spock Framework,
only mock objects created using Spock
can be used with its special syntax
for declaring mock interactions.
Mock objects can be created in the following two equivalent ways:
\autocite[Chapter: Interaction Based Testing - Creating Mock Objects]{SpockFrameworkDoc}
\begin{minted}{groovy}
Subscriber subscriber0 = Mock()
final subscriber1 = Mock(Subscriber)
\end{minted}

\paragraph{Stubbing}
To set up a mock object for stubbing,
the \textit{right shift operator} (\code{<<}) can be used:
\begin{minted}{groovy}
subscriber0.receive("msg") >> true
\end{minted}

This \textit{interaction declaration} instructs the Spock Framework that when
the method \code{receive} is called with
the sole argument \code{"msg"} on
the mock object \code{subscriber0},
the call should return \code{true}.
Interaction declarations for stubbing usually appear in a setup block
to prepare the mock object for usage in the following when or expect block.

The `stub' example in Figure~\ref{fig:SpockTestExample}
also uses the special underscore (\code{_}) identifier.
Used as a method argument, it denotes `any argument'
and thus makes the stub from the example return \code{true}
on every single-argument call to its \code{receive} method.
