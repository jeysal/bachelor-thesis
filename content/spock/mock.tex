\section{Interaction testing}
The Spock Framework comes with an own library
for creating test mocks
(which are at the same time also test stubs).
While it is possible to use any such library
written in Java or another language
that can be compiled to JVM bytecode
in conjunction with the Spock Framework,
only mock objects created using Spock
can be used with its special syntax
for declaring mock interactions.
Mock objects can be created in the following two equivalent ways:
\autocite[Chapter: Interaction Based Testing - Creating Mock Objects]{SpockFrameworkDoc}
\begin{minted}{groovy}
Subscriber subscriber0 = Mock()
final subscriber1 = Mock(Subscriber)
\end{minted}

\paragraph{Stubbing}
To set up a mock object for stubbing,
the \textit{right shift operator} (\code{>>}) can be used:
\begin{minted}{groovy}
subscriber0.receive("msg") >> true
\end{minted}

This \textit{interaction declaration} instructs the Spock Framework that when
the method \code{receive} is called with
the sole argument \code{"msg"} on
the mock object \code{subscriber0},
the call should return \code{true}.
Interaction declarations for stubbing usually appear in a \code{setup} block
to prepare the mock object for usage in the following \code{when} or \code{expect} block.

The `stub' example
in lines 39 to 47 of Figure~\ref{fig:SpockTestExample}
also uses the special underscore (\code{_}) identifier.
Used as a method argument, it denotes `any argument'
and thus makes the stub from the example return \code{true}
on every single-argument call to its \code{receive} method.

\paragraph{Mocking}
To setup up expectations on a mock object,
the \textit{multiplication operator} (\code{*}) can be used as shown
in lines 49 to 56 of Figure~\ref{fig:SpockTestExample}:
\begin{minted}{groovy}
1 * subscriber0.receive("msg")
\end{minted}

This is also an interaction declaration. It asserts that
the method \code{receive} was called \textbf{exactly once} with
the sole argument \code{"msg"} on
the mock object \code{subscriber0}.
Interaction declarations for mocking usually appear in a \code{then} block
to verify that the expected calls to the mock object
have been made in a the preceding \code{when} block.

The underscore placeholder could also be used here,
either as the method argument,
or even as the cardinality
on the left-hand side of the multiplication operation,
which denotes `any number of invocations, including zero'.
That would usually makes it a worthless interaction declaration,
however, it can be useful to allow certain interactions when followed by
a declaration that forbids a superset of those interactions.
\autocite[Chapter: Interaction Based Testing - Mocking - Strict Mocking]{SpockFrameworkDoc}

\paragraph{Advanced interaction declarations}
The interaction testing DSL of the Spock Framework
is capable of much more than the functionality described here,
but we will not implement or need to know about any of
those advanced interaction testing features.
For the sake of completeness
and to suggest some material for future experimentation,
we will briefly mention some of that functionality.

Stubbing and mocking can be combined
within an interaction declaration.
This way, the test can both assert that
a method call has occurred and
stub a return value for that same call.
Declaring the stub and mock interactions separately
will not result in the desired behavior ---
only one interaction declaration will be matched.
\autocite[Chapter: Interaction Based Testing - Combining Mocking and Stubbing]{SpockFrameworkDoc}

Other features include
\autocite[Chapter: Interaction Based Testing]{SpockFrameworkDoc}
\begin{itemize}
  \item \textit{spy} objects,
  \item stubs returning different values for each invocation,
    \begin{itemize}
      \item even dynamically computed ones,
    \end{itemize}
  \item declaring interactions on mock object creation,
  \item order-sensitive interaction expectations,
  \item cardinality ranges in mock interaction declarations,
  \item matching method names using regular expressions, and
  \item various advanced method argument constraints, such as
    \begin{itemize}
      \item matching all arguments \textit{except} a certain value,
      \item matching arguments by type,
      \item matching an arbitrary number of arguments, and
      \item dynamically matching arguments using a closure.
    \end{itemize}
\end{itemize}
