\section{Motivation}
In software development, automated tests are one of the key tools
to achieve a high level of software quality.
Writing tests efficiently itself requires good support
from the programming language, standard library, or third party tools.
The two types of such testing tools we will primarily look at are
\begin{itemize}
  \item \textit{assertion libraries} and
  \item \textit{stubbing / mocking libraries}.
\end{itemize}
Both can be standalone or part of a greater testing framework or utility library.
The former are used to specify a part of the \textit{test oracle},
deciding whether a test case is considered to pass successfully or fail with an error.
The latter are particularly important for substituting subsystems in component testing,
and mock expectations can also form another part of the test oracle.

Assertions are often encoded as chained function calls,
sometimes called \textit{BDD (behavior-driven development)} style,
or in a more classical style as plain calls to an assert function:
\autocite{ChaiBddDoc}\autocite{ChaiTddDoc}
\begin{minted}{javascript}
expect(1).to.equal(1);
assert(1 === 1);
\end{minted}
Stubbing and mocking libraries tend to follow a similar API design approach:
\autocite{SinonStubDoc}\autocite{SinonMockDoc}
\begin{minted}{javascript}
const doubleStub = stub().withArgs(21).returns(42);
const myMock = mock(myApi).expects("method")
                          .once().withArgs(42, 1337);
\end{minted}

Using these libraries to abstract away
the implementation details of our test oracles
is key to allowing test authors to
focus only on the inputs and expectations in a test case.
But the libraries do not eliminate all boilerplate code.
For example, we still need to repeat the \code{assert} call
for each assertion in a test case that contains multiple of them,
which may be a very minor annoyance,
but can still be considered excess boilerplate.

More significantly, stubbing and mocking does not look natural at all.
Why do we need to declare stub and mock calls like
\begin{minted}{javascript}
mock(myApi).expects("method").withArgs(42, 1337)
\end{minted}
instead of the ordinary
\begin{minted}{javascript}
myApi.method(42, 1337)
\end{minted}
that we know from the regular method call syntax as defined by the language?
