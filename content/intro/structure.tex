\section{Structure}
In the following Chapter~\ref{chap:Testing},
we will highlight the aspects of automated software testing
that our later work will touch on,
including test assertions, stubs and mocks.

Afterwards, in Chapter~\ref{chap:Spock},
we will take a closer look at the implementation of DSLs in general
and at the DSL of the Spock Framework in particular.
To explain the very foundation of how the Spock Framework
manages to implement its DSL,
we will work out the basics of
syntax tree traversal and transformation
in Chapter~\ref{chap:Ast}.

In Chapter~\ref{chap:Spockjs},
we will see how the current version of spockjs
applies those techniques in order to
implement functionality similar to the original Spock Framework
for the JavaScript ecosystem,
and which other features it provides.
Chapter~\ref{chap:Interaction}
will then tackle the groundwork that is necessary
to add interaction testing capabilities to spockjs
while supporting the use of multiple mocking libraries.

Chapters~\ref{chap:DirectCompilation}~and~\ref{chap:RuntimeDispatch}
describe two implementations of
the full library-agnostic interaction testing feature set,
following the \textit{direct compilation}
and \textit{runtime dispatch}
approaches, respectively.
In Chapter~\ref{chap:Conclusion},
we wrap up by comparing the approaches,
both with regard to their specific upsides and downsides
in the context of spockjs interaction testing,
and from a more general standpoint of
compile-time versus runtime work.
