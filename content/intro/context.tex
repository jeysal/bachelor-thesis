\section{Context}
The popular testing framework \textit{Spock Framework}
for the Groovy programming language takes a route
that significantly differs from the usage of
classic testing helper libraries.
It defines a \textit{domain-specific language (DSL)} for the test cases
that allows expressing assertions and stub / mock interactions in a very concise way.
This DSL cannot be implemented entirely by a `userspace' library, therefore,
Spock hooks into the compiler and transforms the syntax tree prior to test execution.

Spock allows expressing assertions in \code{expect} blocks,
or alternatively in \code{when}-\code{then} blocks:
\autocite{SpockFrameworkDoc}
\begin{minted}{groovy}
expect:
Math.max(1, 2) == 2
// ---
when:
stack.push(elem)
then:
stack.size() == 1
\end{minted}
Stub and mock interactions can be expressed in the same fashion:
\autocite{SpockFrameworkDoc}
\begin{minted}{groovy}
then:
// return "ok" for every call to receive
subscriber.receive(_) >> "ok"
// expect 1 call to receive with argument "hello"
1 * subscriber2.receive("hello")
\end{minted}

\textit{Spockjs} is a work-in-progress implementation
of Spock-style testing for JavaScript.
\autocite{SpockjsGithub}
The latest version at the time of writing
supports use of the \code{expect} and \code{when}-\code{then}
labels to express assertions,
but has not yet implemented any
of the more advanced features of Spock.
Spockjs is test runner-agnostic by
hooking into the popular JavaScript compiler babel,
which integrates with most modern test runners.
\autocite{BabelSetupDoc}
