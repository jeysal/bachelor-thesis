\chapter{Abstract syntax trees}
An \textit{abstract syntax tree (AST)} represents
the syntactic structure of program source code.
It is a datastructure used to
store source code internally by various tools,
such as compilers, code formatters, and
integrated development environments.
It can be built from the source code tokens using a \textit{parser}.
Unlike a \textit{concrete syntax tree},
an abstract syntax tree contains insufficient information
to perform the inverse operation and
recreate the original source code from the tree.
\autocite{AstImplIdioms}

Its \textit{vertices (nodes)} each store a node type,
which can be discriminated upon
in an implementation language-specific manner.
Common examples for node types include
\begin{itemize}
  \item \textit{binary expressions},
  \item \textit{identifiers}, and
  \item \textit{literals}.
\end{itemize}
Vertices can also store additional attributes,
depending on their type.
Examples of additional attributes include
\begin{itemize}
  \item the \textit{operator} used in a binary expression
  (although some AST formats instead use different node types
  for different binary operations \autocite{AstImplIdioms}),
  \item the \textit{name} of an identifier, and
  \item the \textit{value} of a literal.
\end{itemize}

The edges are labeled to indicate
the relation between parent and child nodes,
with the parent node type determining
which edge labels are allowed.
Examples of edge labels include
\begin{itemize}
  \item the \textit{test} of an if statement,
  \item its \textit{consequent}, and
  \item its optional \textit{alternate} (`else block')
\end{itemize}
There may also be multiple ordered children
connected to their parent with edges of the same label,
such as the \textit{body} label
for a \textit{block statement} parent.

\section{ESTree}
We will use the ESTree AST format whenever we operate on ASTs.
ESTree is the de facto community standard for ECMAScript ASTs,
used in various popular tools, including
the parsers `Acorn' and `Esprima',
the linter ESLint, and
the compiler Babel,
which we will work with extensively.
\autocite{EstreeSpec}

The ESTree specification contains an interface for each node type,
as well as abstract interfaces that others inherit from.
All node type interfaces inherit --- directly or indirectly ---
from the abstract \code{Node} interface: \autocite{EstreeSpec}
\begin{minted}{javascript}
interface Node {
  type: string;
  loc: SourceLocation | null;
}
\end{minted}
The \code{loc} attribute contains information about
the position of the source code section
that corresponds to the AST node,
but we can disregard it for now.
The \code{type} attribute serves as the discriminator,
with a unique value assigned to each node type
that can be used to determine
the concrete node type of a given node object.

The \code{IfStatement} is an example
of such a concrete node type:
\autocite{EstreeSpec}
\begin{minted}{javascript}
interface IfStatement <: Statement {
    type: "IfStatement";
    test: Expression;
    consequent: Statement;
    alternate: Statement | null;
}
\end{minted}
An instance of it
with its corresponding source code
is shown in Figure~\ref{fig:EstreeIfStmt}.
\begin{figure}
  Source code:
  \begin{minted}[linenos]{javascript}
if (1 === 2) {
  console.log(42);
} else {
  console.log(1337);
}
  \end{minted}
	ESTree node:
  \begin{minted}[linenos]{json}
{
  "type": "IfStatement",
  "start": 0,
  "end": 65,
  "loc": {
    "start": {
      "line": 1,
      "column": 0
    },
    "end": {
      "line": 5,
      "column": 1
    }
  },
  "test": {
    "type": "BinaryExpression"
  },
  "consequent": {
    "type": "BlockStatement"
  },
  "alternate": {
    "type": "BlockStatement"
  }
}
  \end{minted}
  \caption{
    An if statement in source code and its
    ESTree node, serialized as JSON.
    Its child nodes have been shortened;
    they include only the node types.
  }\label{fig:EstreeIfStmt}
\end{figure}

