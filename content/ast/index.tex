\chapter{Abstract syntax trees}
An \textit{abstract syntax tree (AST)} represents
the syntactic structure of program source code.
It is a datastructure used to
store source code internally by various tools,
such as compilers, code formatters, and
integrated development environments.
It can be built from the source code tokens using a \textit{parser}.
Unlike a \textit{concrete syntax tree},
an abstract syntax tree contains insufficient information
to perform the inverse operation and
recreate the original source code from the tree.
\autocite{AstImplIdioms}

Its \textit{vertices (nodes)} each store a node type,
which can be discriminated upon
in an implementation language-specific manner.
Common examples for node types include
\begin{itemize}
  \item \textit{binary expressions},
  \item \textit{identifiers}, and
  \item \textit{literals}.
\end{itemize}
Vertices can also store additional attributes,
depending on their type.
Examples of additional attributes include
\begin{itemize}
  \item the \textit{operator} used in a binary expression
  (although some AST formats instead use different node types
  for different binary operations \autocite{AstImplIdioms}),
  \item the \textit{name} of an identifier, and
  \item the \textit{value} of a literal.
\end{itemize}

The edges are labeled to indicate
the relation between parent and child nodes,
with the parent node type determining
which edge labels are allowed.
Examples of edge labels include
\begin{itemize}
  \item the \textit{test} of an if statement,
  \item its \textit{consequent}, and
  \item its optional \textit{alternate} (`else block')
\end{itemize}
There may also be multiple ordered children
connected to their parent with edges of the same label,
such as the \textit{body} label
for a \textit{block statement} parent.
