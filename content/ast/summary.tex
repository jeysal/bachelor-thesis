\section{Summary}
In this chapter,
we have looked at how a program can
be represented as an abstract syntax tree (AST)
and what parts this type of data structure consists of.
We have learned how to encode ASTs as concrete objects
according to the ESTree specification,
which will be useful when we work with the compiler Babel.

We have also examined some patterns
that can be used to execute different pieces of code
based on the types of tree nodes
that we encounter during tree traversal
and have found the visitor pattern
to be particularly useful for this task.
Finally, we have looked at some of
the tree mutation operations
that are commonly performed
in visitors during tree traversal.
