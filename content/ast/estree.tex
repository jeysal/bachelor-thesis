\section{ESTree}
We will use the ESTree AST format whenever we operate on ASTs.
ESTree is the de facto community standard for ECMAScript ASTs,
used in various popular tools, including
the parsers `Acorn' and `Esprima',
the linter ESLint, and
the compiler Babel,
which we will work with extensively.
\autocite{EstreeSpec}

The ESTree specification contains an interface for each node type,
as well as abstract interfaces that others inherit from.
All node type interfaces inherit --- directly or indirectly ---
from the abstract \code{Node} interface: \autocite{EstreeSpec}
\begin{minted}{javascript}
interface Node {
  type: string;
  loc: SourceLocation | null;
}
\end{minted}
The \code{loc} attribute contains information about
the position of the source code section
that corresponds to the AST node,
but we can disregard it for now.
The \code{type} attribute serves as the discriminator,
with a unique value assigned to each node type
that can be used to determine
the concrete node type of a given node object.

The \code{IfStatement} is an example
of such a concrete node type:
\autocite{EstreeSpec}
\begin{minted}{javascript}
interface IfStatement <: Statement {
    type: "IfStatement";
    test: Expression;
    consequent: Statement;
    alternate: Statement | null;
}
\end{minted}
An instance of it
with its corresponding source code
is shown in Figure~\ref{fig:EstreeIfStmt}.
\begin{figure}
  Source code:
  \begin{minted}{javascript}
if (1 === 2) {
  console.log(42);
} else {
  console.log(1337);
}
  \end{minted}
	ESTree node:
  \begin{minted}{json}
{
  "type": "IfStatement",
  "start": 0,
  "end": 65,
  "loc": {
    "start": {
      "line": 1,
      "column": 0
    },
    "end": {
      "line": 5,
      "column": 1
    }
  },
  "test": {
    "type": "BinaryExpression"
  },
  "consequent": {
    "type": "BlockStatement"
  },
  "alternate": {
    "type": "BlockStatement"
  }
}
  \end{minted}
  \caption{
    An ESTree node for an if statement, serialized as JSON.
    Its child nodes have been shortened; they include only the node types.
  }\label{fig:EstreeIfStmt}
\end{figure}
