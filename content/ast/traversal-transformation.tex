\section{Traversal and transformation}
In the following, we will illustrate
how an AST can be traversed and transformed
by defining a simple example transformation.
We will swap the sides of each
\textit{strict equality comparison} (\code{===})
that occurs in the tree.
The node objects that we operate on
are shaped as defined by the ESTree specification.

\subsection{Traversal}
In object-oriented languages,
different tree node types are represented as
different classes that directly or indirectly
implement the \code{Node} interface.
There are multiple ways to navigate the tree,
each resulting in a distinct order of node occurence.
The most common algorithms for tree traversal in general are
\begin{itemize}
  \item \textit{Depth-first search (DFS)},
  which can be performed
  \begin{itemize}
    \item \textit{Pre-order},
    \item \textit{In-order}, or
    \item \textit{Post-order}, and
  \end{itemize}
  \item \textit{Breadth-first search (BFS)}.
\end{itemize}
However, we can disregard the traversal order
for our example transformation
(and for the other transformations in this paper),
because we will only modify single nodes
regardless of the context they appear in,
and we do not hold any state across nodes.

Instead, we are interested in the structure of
the code that manipulates AST nodes.
There are multiple ways to switch between
different actions depending on the
type of the node we are currently processing.

\paragraph{Type switching}
The primitive approach shown in Figure~\ref{fig:TreeNodeTypeSwitching}
is to simply switch between code paths depending on the node type,
which we can retrieve from the \code{type} property
--- or by using an \code{instanceof} operator
with many other languages and AST formats.

\begin{figure}
  \begin{minted}{javascript}
switch (node.type) {
  case 'Identifier':
    console.log(node.name);
    break;
  case 'Literal':
    console.log(node.value);
    break;
  // ...
  default:
    throw new Error(`unknown node type '${node.type}'`);
}
  \end{minted}
  \caption{
    Switching between different node types.
  }\label{fig:TreeNodeTypeSwitching}
\end{figure}

However, this approach is generally considered bad design
in the world of \textit{object-oriented programming (OOP)},
\autocite{VisitorPatternEssence}
requires type casts in languages that are
neither dynamic nor structurally-typed, and
needs a fallback case instead of
verifying that all types are covered at compile time.

\paragraph{Abstract handler methods}
An alternative approach that better adheres to
\textit{object-oriented design (OOD)} principles
is to define the handler method in the \code{Node} interface
as shown in Figure~\ref{fig:TreeNodeAbstractHandlerMethods}.
\autocite{VisitorPatternEssence}
We implement it in each node type class
and let the language dynamically dispatch for us.

\begin{figure}
  \begin{minted}{typescript}
interface Node {
  handle(): void;
}

class Identifier implements Node {
  handle() {
    console.log(this.name);
  }
}

class Literal implements Node {
  handle() {
    console.log(this.value);
  }
}
  \end{minted}
  \caption{
    Defining handlers on each node type.
  }\label{fig:TreeNodeAbstractHandlerMethods}
\end{figure}

The major disadvantage of this approach is that
all handlers must be defined on the node types themselves.
This becomes impractical if we want to
apply multiple transformations to the AST,
each with their own node handler logic,
or even if we just want to define multiple handlers
for totally different operations on the tree,
such as machine code generation, bytecode generation,
tree serialization, and pretty-printing.

\paragraph{Visitor pattern}
The final and most popular approach is
to employ the \textit{visitor pattern}
as shown in Figure~\ref{fig:TreeNodeVisitorPattern}.
\autocite{DesignPatterns}
After dispatching to the \code{accept} method
of the current node type,
we further delegate to the \textit{visitor} object,
which provides a method to handle each node type,
including the one for the now well-known type
of the current node \code{this}.

\begin{figure}
  \begin{minted}{typescript}
interface Node {
  accept(v: Visitor): void;
}

class Identifier implements Node {
  accept(v: Visitor) {
    v.visitIdentifier(this);
  }
}

class Literal implements Node {
  handle(v: Visitor) {
    v.visitLiteral(this);
  }
}


interface Visitor {
  visitIdentifier(identifier: Identifier): void;
  visitLiteral(literal: Literal): void;
}

class LogVisitor implements Visitor {
  visitIdentifier(identifier: Identifier) {
    console.log(identifier.name);
  }

  visitLiteral(literal: Literal) {
    console.log(identifier.value);
  }
}
  \end{minted}
  \caption{
    The visitor pattern.
  }\label{fig:TreeNodeVisitorPattern}
\end{figure}

This approach is both immaculate OOD and
fulfills the requirement to
decouple the node types from their handlers,
of which there may be a large quantity,
serving different purposes.
While traversing the tree we can now apply each visitor
(or a list of visitors) to each node.

\subsection{Transformation}
Our strict equality swap visitor will only apply to \code{BinaryExpression}s,
of which strict equality comparison expressions are a subset.
We filter any other kind of binary operation;
only for strict equality comparisons we swap the
left subexpression with the right one:
\begin{minted}{javascript}
if (node.operator === '===') {
  [node.left, node.right] = [node.right, node.left];
}
\end{minted}

Other common transformations include
\autocite{BabelPluginHandbook}
\begin{itemize}
  \item replacing a node with a new node
    (that is still valid with regard to its parent),
    \begin{itemize}
      \item in particular,
        wrapping a node into another node
        by replacing it with a new node
        that has the old one as its child,
    \end{itemize}
  \item inserting a sibling node,
  \item removing a node, and
  \item renaming a binding.
\end{itemize}
