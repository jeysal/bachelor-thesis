\chapter*{Appendix}
\addcontentsline{toc}{chapter}{Appendix}
\pagestyle{plain}
\setminted{samepage=no,fontsize=\ssmall}

% TODO this chapter contains the code that is directly relevant to the paper. explain section titles. some words on where to find the full code with a better reading experience...

\section*{experimental-mocking-direct-compilation}

\subsection*{packages/babel-plugin-spock/src/index.ts}
\begin{minted}{typescript}
import { PluginObj } from '@babel/core';
import { NodePath } from '@babel/traverse';
import * as BabelTypes from '@babel/types';

import { extractConfigFromState } from '@spockjs/config';

import assertifyStatement, {
  labels as assertionBlockLabels,
} from '@spockjs/assertion-block';
import transformInteractionDeclarationStatement, {
  declarationLabels as interactionDeclarationLabels,
  verificationLabel as interactionVerificationLabel,
} from '@spockjs/interaction-block';

const transformLabeledBlockOrSingle = (
  transform: (statementPath: NodePath<BabelTypes.Statement>) => void,
  path: NodePath<BabelTypes.LabeledStatement>,
): void => {
  const labeledBodyPath = path.get('body') as NodePath<BabelTypes.Statement>;

  switch (labeledBodyPath.type) {
    case 'BlockStatement':
      // power-assert may add statements in betweeen,
      // so never reuse body array
      const statementPaths = () =>
        (labeledBodyPath.get('body') as any) as NodePath<
          BabelTypes.Statement
        >[];

      statementPaths().forEach(transform);

      // remove label
      path.replaceWithMultiple(statementPaths().map(stmtPath => stmtPath.node));
      break;
    default:
      transform(labeledBodyPath);

      // remove label
      if (!path.removed) {
        path.replaceWith(labeledBodyPath);
      }
  }
};

export default (babel: { types: typeof BabelTypes }): PluginObj => ({
  visitor: {
    LabeledStatement(path, state) {
      const config = extractConfigFromState(state);
      const label = path.node.label.name;

      // assertion block
      if (assertionBlockLabels.includes(label)) {
        transformLabeledBlockOrSingle(
          assertifyStatement(babel, state, config),
          path,
        );
      }

      // interaction block
      else if (interactionDeclarationLabels.includes(label)) {
        transformLabeledBlockOrSingle(
          transformInteractionDeclarationStatement(babel.types, config).declare,
          path,
        );
      } else if (label === interactionVerificationLabel) {
        transformLabeledBlockOrSingle(
          transformInteractionDeclarationStatement(babel.types, config).verify,
          path,
        );
      }
    },
  },
});
\end{minted}

\subsection*{packages/config/src/hooks/interaction.ts}
\begin{minted}{typescript}
import * as BabelTypes from '@babel/types';

import { InternalConfig } from '..';

export interface BaseInteractionDeclaration {
  mockObject: BabelTypes.Expression;
  args: (BabelTypes.Expression | BabelTypes.SpreadElement)[];
}
export interface MockInteractionDeclaration extends BaseInteractionDeclaration {
  kind: 'mock';
  cardinality: BabelTypes.Expression;
}
export interface StubInteractionDeclaration extends BaseInteractionDeclaration {
  kind: 'stub';
  returnValue: BabelTypes.Expression;
}
export interface CombinedInteractionDeclaration
  extends BaseInteractionDeclaration {
  kind: 'combined';
  cardinality: BabelTypes.Expression;
  returnValue: BabelTypes.Expression;
}
export type InteractionDeclaration =
  | MockInteractionDeclaration
  | StubInteractionDeclaration
  | CombinedInteractionDeclaration;

export type InteractionProcessor = (
  t: typeof BabelTypes,
  config: InternalConfig,
) => {
  /**
   * Whether the processor is meant for standalone use (`true`),
   * as is the case for those implementing mocking library bindings,
   * or it is an auxiliary processor (`false`),
   * such as one that provides better integration with a test runner.
   * Users will see an error when attempting to use interaction blocks
   * without a primary interaction processor.
   */
  primary: boolean;
  declare(interaction: InteractionDeclaration): BabelTypes.Statement;
  verify(mockObject: BabelTypes.Expression): BabelTypes.Statement;
};
\end{minted}

\subsection*{packages/interaction-block/src/parser.ts}
\begin{minted}{typescript}
import { NodePath } from '@babel/traverse';
import { BinaryExpression, Expression, Node } from '@babel/types';
import {
  BaseInteractionDeclaration,
  CombinedInteractionDeclaration,
  InteractionDeclaration,
  MockInteractionDeclaration,
  StubInteractionDeclaration,
} from '@spockjs/config';

const mockOperator = '*';
const stubOperator = '>>';

type InteractionDeclarationParser<
  E extends Node,
  I extends BaseInteractionDeclaration
> = (expressionPath: NodePath<E>) => I;

const parseCall: InteractionDeclarationParser<
  Expression,
  BaseInteractionDeclaration
> = (expressionPath: NodePath) => {
  if (expressionPath.isCallExpression()) {
    const {
      node: { callee: mockObject, arguments: args },
    } = expressionPath;
    return {
      mockObject,
      args,
    };
  }
  throw expressionPath.buildCodeFrameError(
    `Expected a call expression, but got an expression of type ${
      expressionPath.type
    }`,
  );
};

const parseMockInteractionDeclaration: InteractionDeclarationParser<
  BinaryExpression,
  MockInteractionDeclaration
> = expressionPath => ({
  ...parseCall(expressionPath.get('right') as NodePath<Expression>),
  kind: 'mock',
  cardinality: expressionPath.node.left,
});

const parseStubInteractionDeclaration: InteractionDeclarationParser<
  BinaryExpression,
  StubInteractionDeclaration | CombinedInteractionDeclaration
> = expressionPath => {
  const leftPath = expressionPath.get('left') as NodePath<Expression>;
  const {
    node: { right: returnValue },
  } = expressionPath;

  if (leftPath.isBinaryExpression()) {
    const {
      node: { operator },
    } = leftPath;
    if (operator === mockOperator) {
      return {
        ...parseMockInteractionDeclaration(leftPath),
        returnValue,
        kind: 'combined',
      };
    }
    throw new Error(
      `Expected operator '${mockOperator}' (for combined mocking and stubbing), ` +
        `but got operator '${operator}'`,
    );
  }

  return {
    ...parseCall(leftPath),
    returnValue,
    kind: 'stub',
  };
};

const parseInteractionDeclaration: InteractionDeclarationParser<
  BinaryExpression,
  InteractionDeclaration
> = expressionPath => {
  const { node: expression } = expressionPath;
  const { operator } = expression;

  switch (operator) {
    case mockOperator:
      return parseMockInteractionDeclaration(expressionPath);
    case stubOperator:
      return parseStubInteractionDeclaration(expressionPath);
    default:
      throw expressionPath.buildCodeFrameError(
        `Expected operator '${mockOperator}' (for mocking) ` +
          `or '${stubOperator}' (for stubbing), ` +
          `but got operator '${operator}'`,
      );
  }
};

export default parseInteractionDeclaration;
\end{minted}
