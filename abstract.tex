\begin{abstract}
  \addcontentsline{toc}{chapter}{Abstract}

  {\noindent\large\textbf{Abstract}\par}\noindent
  \textit{Spockjs} is a testing tool
  that integrates with most modern JavaScript test runners
  and allows users to write their tests
  using the declarative `block' style popularized by
  the \textit{Spock Framework} for Groovy tests.
  Adding to the \textit{assertion blocks}
  already implemented in the
  \textit{domain-specific language} of spockjs,
  we develop another Spock Framework-inspired
  block type: \textit{Interaction blocks}.
  To implement this new feature
  in a mocking library-agnostic way,
  we experiment with and evaluate two different approaches
  to distributing the work between
  compile time and runtime.

  \vspace{3cm}
  {\noindent\large\textbf{German Abstract}\par}\noindent
  \textit{Spockjs} ist ein Testing-Tool,
  das mit den meisten modernen JavaScript Test-Runnern verwendet werden kann
  und es Nutzern ermöglicht, ihre Tests
  in dem deklarativen `Block'-Stil des \textit{Spock Framework}s für Groovy-Tests zu schreiben.
  Zusätzlich zu den \textit{Assertion-Blöcken},
  die in der \textit{domänenspezifischen Sprache} von spockjs bereits implementiert sind,
  wird ein weiterer durch das Spock Framework inspirierter
  Blocktyp entwickelt: \textit{Interaction-Blöcke}.
  Um diese neue Funktionalität zu implementieren
  und dabei von der der verwendeten Mocking-Library unabhängig zu bleiben,
  wird mit zwei verschiedenen Ansätzen
  zur Verteilung der Arbeit zwischen Compile-Zeit und Laufzeit experimentiert
  und diese Ansätze werden bewertet.
\end{abstract}
